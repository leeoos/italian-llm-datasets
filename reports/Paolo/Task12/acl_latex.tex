% This must be in the first 5 lines to tell arXiv to use pdfLaTeX, which is strongly recommended.
\pdfoutput=1
% In particular, the hyperref package requires pdfLaTeX in order to break URLs across lines.

\documentclass[11pt]{article}

% Change "review" to "final" to generate the final (sometimes called camera-ready) version.
% Change to "preprint" to generate a non-anonymous version with page numbers.
\usepackage[final]{acl}

% Standard package includes
\usepackage{times}
\usepackage{latexsym}

% For proper rendering and hyphenation of words containing Latin characters (including in bib files)
\usepackage[T1]{fontenc}
% For Vietnamese characters
% \usepackage[T5]{fontenc}
% See https://www.latex-project.org/help/documentation/encguide.pdf for other character sets

% This assumes your files are encoded as UTF8
\usepackage[utf8]{inputenc}

% This is not strictly necessary, and may be commented out,
% but it will improve the layout of the manuscript,
% and will typically save some space.
\usepackage{microtype}

% This is also not strictly necessary, and may be commented out.
% However, it will improve the aesthetics of text in
% the typewriter font.
\usepackage{inconsolata}

%Including images in your LaTeX document requires adding
%additional package(s)
\usepackage{graphicx}

% If the title and author information does not fit in the area allocated, uncomment the following
%
%\setlength\titlebox{<dim>}
%
% and set <dim> to something 5cm or larger.

\title{Task 12 TAG\_it}

% Author information can be set in various styles:
% For several authors from the same institution:
% \author{Author 1 \and ... \and Author n \\
%         Address line \\ ... \\ Address line}
% if the names do not fit well on one line use
%         Author 1 \\ {\bf Author 2} \\ ... \\ {\bf Author n} \\
% For authors from different institutions:
% \author{Author 1 \\ Address line \\  ... \\ Address line
%         \And  ... \And
%         Author n \\ Address line \\ ... \\ Address line}
% To start a separate ``row'' of authors use \AND, as in
% \author{Author 1 \\ Address line \\  ... \\ Address line
%         \AND
%         Author 2 \\ Address line \\ ... \\ Address line \And
%         Author 3 \\ Address line \\ ... \\ Address line}

\author{Paolo Renzi \\
  Sapienza Università di Roma\\\
  }

%\author{
%  \textbf{First Author\textsuperscript{1}},
%  \textbf{Second Author\textsuperscript{1,2}},
%  \textbf{Third T. Author\textsuperscript{1}},
%  \textbf{Fourth Author\textsuperscript{1}},
%\\
%  \textbf{Fifth Author\textsuperscript{1,2}},
%  \textbf{Sixth Author\textsuperscript{1}},
%  \textbf{Seventh Author\textsuperscript{1}},
%  \textbf{Eighth Author \textsuperscript{1,2,3,4}},
%\\
%  \textbf{Ninth Author\textsuperscript{1}},
%  \textbf{Tenth Author\textsuperscript{1}},
%  \textbf{Eleventh E. Author\textsuperscript{1,2,3,4,5}},
%  \textbf{Twelfth Author\textsuperscript{1}},
%\\
%  \textbf{Thirteenth Author\textsuperscript{3}},
%  \textbf{Fourteenth F. Author\textsuperscript{2,4}},
%  \textbf{Fifteenth Author\textsuperscript{1}},
%  \textbf{Sixteenth Author\textsuperscript{1}},
%\\
%  \textbf{Seventeenth S. Author\textsuperscript{4,5}},
%  \textbf{Eighteenth Author\textsuperscript{3,4}},
%  \textbf{Nineteenth N. Author\textsuperscript{2,5}},
%  \textbf{Twentieth Author\textsuperscript{1}}
%\\
%\\
%  \textsuperscript{1}Affiliation 1,
%  \textsuperscript{2}Affiliation 2,
%  \textsuperscript{3}Affiliation 3,
%  \textsuperscript{4}Affiliation 4,
%  \textsuperscript{5}Affiliation 5
%\\
%  \small{
%    \textbf{Correspondence:} \href{mailto:email@domain}{email@domain}
%  }
%}

\begin{document}
\maketitle

\section{The description of the dataset}

The TAG-it dataset, originates from the EVALITA 2020 competition, aims to explore the relationships among gender, 
age, and various topics across different blog authors. Originally, the dataset was designed to capture these interactions, 
with a focus on longer texts from blog posts. However, for the evaluation task, a modified version of the dataset 
is employed, concentrating solely on shorter posts to accurately classify the topics of grouped posts. 
This approach allows for a more focused examination of the relationship between author characteristics and topic 
classification.

\section{Methodology to reframe the dataset}

I exploited the fact that each row of data ends with the name of the dataset and used that to separate each line. 
In the test set the data is taken from two datasets so the name changes accordingly but i still used the same approch 
by first dividing for political\_test and then I iterated in each line to divide it again by Religious\_test and put it all 
in a list. I then iterated over the resulting list and created a dictionary where the keys are the id of the tweet and 
the values are themselfs a dictionary with as key the type of data (Ex. "text", "label", "created at") and the values is 
the data. I differentiate between test and training and textual data and conte data from the filepath string.

\section{Prompts}

I created 4 prompts for each task, and i added in task 2 the following metadata, created at and user\_created\_at beacause the 
age of the profile at the time of writing might have a correlation with the task. For the same reason I added status, friend
e followers count.

\section{How to run}
To run the parser you should download all the required libraries (Gitpython zipfie and pandas) set the folder contating 
task\_12.py as CWD and then run
\begin{quote}
  \begin{verbatim}
    python task_12.py
  \end{verbatim}
\end{quote}
\noindent You will need to download the dataset manually from this link https://live.european-language-grid.eu/catalogue/corpus/8112/download/,
and put it in the same directory as the .py file
\begin{table*}
  \centering
  \caption{Distractor Relationships}
  \label{tab:distractor}
  \begin{tabular}{lll}
  \hline
  \textbf{Topic}           & \textbf{Related Topics}                                                                                                                         \\ \hline
  CELEBRITÀ                & MEDICINA-ESTETICA, INTRATTENIMENTO                                                                                                                   \\ \hline
  ANIME                    & GIOCHI, GIOCHI\_DI\_RUOLO, INTRATTENIMENTO                                                                                                           \\ \hline
  FUMO                     & TECNOLOGIA, OROLOGI                                                                                                                                  \\ \hline
  AUTO-MOTO                & SPORT, MOTO                                                                                                                                           \\ \hline
  SPORT                    & AUTO-MOTO, MOTO, GIOCHI                                                                                                                              \\ \hline
  MOTO                     & SPORT, AUTO-MOTO                                                                                                                                      \\ \hline
  METAL-DETECTING          & NATURA, TECNOLOGIA, AUTO-MOTO                                                                                                                         \\ \hline
  TECNOLOGIA               & INTRATTENIMENTO, GIOCHI, AUTO-MOTO                                                                                                                    \\ \hline
  MEDICINA-ESTETICA        & CELEBRITÀ                                                                                                                                             \\ \hline
  INTRATTENIMENTO          & GIOCHI, GIOCHI\_DI\_RUOLO, SPORT, ANIME                                                                                                               \\ \hline
  NATURA                   & METAL-DETECTING                                                                                                                                       \\ \hline
  GIOCHI                   & GIOCHI\_DI\_RUOLO, INTRATTENIMENTO                                                                                                                    \\ \hline
  GIOCHI\_DI\_RUOLO        & GIOCHI, INTRATTENIMENTO                                                                                                                               \\ \hline
  OROLOGI                  & CELEBRITÀ                                                                                                                                             \\ \hline
  \end{tabular}
  \caption{Example commands for accented characters, to be used in, \emph{e.g.}, Bib\TeX{} entries.}
\end{table*}
 but .


\end{document}
